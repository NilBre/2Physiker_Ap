Im Vakuum gilt $\rho = 0$ und $j=0$ , also lauten die Maxwellgleichungen
nun
\begin{align}
  \nabla \cdot \symbf{E} &= 0\\
  \nabla \cdot \symbf{B} &= 0\\
  \nabla \times \symbf{E} &= \partial_{t} \symbf{B}
        \label{eqn:max3}\\
  \nabla \times \symbf{B} &= \mu_{0} \epsilon_{0}
  \partial_{t} \symbf{E}
\intertext{
Außerdem benutzen wir, dass $\mu_{0} \epsilon_{0}=\frac{1}{c^2}$ gilt.
Somit ergibt sich, wenn man die Rotation auf \eqref{eqn:max3}
anwendet}
  \nabla \times \nabla \times \symbf{E} &= \nabla \times
  \left (\partial_{t} \symbf{B} \right)
\intertext{
Nun kann man im nächsten Schritt die Graßmann Identität
benutzen und außerdem auch
$\nabla \cdot \symbf{E} = 0$
wie nach Vorraussetzung.
Zusätzlich können nach dem Satz von Schwarz
\footnote{Das heißt, die Reihenfolge der Ableitungen darf vertauscht werden}
die Ableitungen
vertauscht werden.}
  \nabla \cdot \left(\nabla \cdot \symbf{E} \right)-
  \triangle \symbf{E} &= \partial_{t}
  \left(\nabla \times \symbf{B} \right)\\
  - \triangle \symbf{E} &= \partial_{t} \left (\mu_{0} \epsilon_{0}
  \partial_{t} \symbf{E} \right)\\
  \frac{\partial^2}{\partial t^2} \mu_{0} \epsilon_{0} \symbf{E} -
  \triangle \symbf{E} &= 0 \\
  \symbf{E} \left(\frac{\partial^2}{\partial t^2}
  \frac{1}{c^2}-\triangle \right) &= 0\\
  \left(\frac{\partial^2}{\partial t^2}
  \frac{1}{c^2}-\triangle \right) \coloneq \square\\
  \square \symbf{E} &= 0
\end{align}
Die Wellengleichung für das Magnetfeld werden analog
hergeleitet.
