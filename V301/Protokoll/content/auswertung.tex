\section{Auswertung}
\label{sec:Auswertung}

Die direkt aufgenommenen Werte für die Leerlaufspannung $U_0$ und den Innenwiderstand $R_V$ lauten:

\begin{align*}
  U_0 &= 1,4 \, \si{\volt} \\
  R_V &\approx 10 \, \si{\mega\ohm}
\end{align*}

\noindent Um den Innenwiderstand $R_i$ und die Leerlaufspannung $U_0$ der Monozelle bestimmen zu können, werden die gemessenen Werte für $U_k$ gegen \textbf{I} aufgetragen.
Dieser Plot ist in Abbildung \ref{fig:klemmspannung} zu sehen.
Mit Python wurd eine lineare Ausgleichsrechnung nach $f = ax + b$ vorgenommen.
Es ergeben sich nach Gleichung \eqref{eqn:uk} für die Ausgleichsrechnung folgende Werte:

\begin{align*}
  -a &= R_i = 14,67 \pm 0,28 \, \si{\ohm} \\
  b &= U_0 = 1,35 \pm 0,01 \, \si{\volt}
\end{align*}

\noindent Die zugrunde liegenden Messwerte können Tabelle \ref{tab:klemmspannung} entnommen werden.

\begin{table}
\centering
\caption{Klemmenspannung $U_k$ in Abhängigkeit von Belastungsstrom $I$ und Belastungswiderstand $R$}
\label{tab:klemmspannung}
\sisetup
{table-format=1.2}
\begin{minipage}{4.5cm}
\begin
{tabular}{S[table-format=3.5] S[table-format=3.2]}
\toprule
{$I/\,\si{mA}$} &{$U_k/\,\si{V}$} \\
\midrule
86 & 0.05\\
83 & 0.13\\
67 & 0.38\\
61.5 & 0.47\\
54 & 0.59\\
48 & 0.69\\
43 & 0.71\\
37 & 0.77\\
34.5 & 0.80\\
33 & 0.87\\
30 & 0.90\\
28 & 0.92\\
27 & 1.00\\
26 & 0.96 \footnote{Skalenwechsel am Voltmeter}\\
25 & 0.99\\
23.5 & 0.99\\
22 & 1.02\\
21 & 1.05\\
20 & 1.05\\
19 & 1.08\\
19 & 1.08\\
\bottomrule
\end{tabular}
\par
\vspace{-0.75\skip\footins}
\renewcommand{\footnoterule}{}
\end{minipage}
\end{table}

\begin{figure}[h]
  \centering
  \includegraphics{plot.pdf}
  \caption{Messwerte der Monozelle ohne Gegenspannung mit linearer Ausgleichsgerade}
  \label{fig:klemmspannung}
\end{figure}

\noindent Eine ähnliche Rechnung wird für den Versuch mit der Gegenspannungsquelle (vgl. Abbildung \ref{fig:aufbaub}) wiederholt.
Allerdings muss bei der Bestimmung von $R_i$ und $U_0$ beachtet werden, dass sich die Stromrichtung durch die Gegenspannungsquelle umkehrt.
Gleichung \eqref{eqn:uk} verändert sich daher zu:

\begin{equation}
  U_k = U_0 + I R_i
\end{equation}

\noindent Die Messdaten aus Tabelle \ref{tab:klemmspannung_gegenspannung} werden dafür in Abbildung \ref{fig:klemmspannung_gegenspannung} gegeneinander aufgetragen.
Auch hier wird eine lineare Ausgleichsrechnung vorgenommen.
Es ergeben sich dabei folgende Werte:

\begin{align*}
  a &= R_i = 10,71 \pm 0,41 \, \si{\ohm} \\
  b &= U_0 = 1,64 \pm 0,03 \, \si{\volt}
\end{align*}

\begin{table}[H]
\centering
\caption{Abhängigkeiten mit angelegter Gegenspannung}
\label{tab:klemmspannung_gegenspannung}
\sisetup
{table-format=1.2}
\begin{minipage}{4.5cm}
\begin
{tabular}{S[table-format=3.0] S[table-format=3.5] S[table-format=3.2]}
\toprule
{$R/\,\si{\Omega}$} &{$I/\,\si{mA}$} &{$U_k/\,\si{V}$} \\
\midrule
0 & 170.0 & 3.5\\
2.5 & 160.0 & 3.4\\
5.0 & 140.0 & 3.1\\
7.5 & 130.0 & 2.9\\
10.0 & 120.0 & 2.8\\
12.5 & 76.0 \footnote{Skalenwechsel Amperemeter} & 2.6\\
15.0 & 71.0 & 2.55 \footnote{Skalenwechsel Voltmeter}\\
17.5 & 65.0 & 2.43\\
20.0 & 61.5 & 2.37\\
22.5 & 57.5 & 2.31\\
25.0 & 54.5 & 2.25\\
27.5 & 51.0 & 2.22\\
30.0 & 48.5 & 2.16\\
32.5 & 47.0 & 2.13\\
35.0 & 45.0 & 2.10\\
37.5 & 43.0 & 2.07\\
40.0 & 41.0 & 2.04\\
42.5 & 39.0 & 2.01\\
45.0 & 38.0 & 1.98\\
47.5 & 37.0 & 1.98\\
50.0 & 36.0 & 1.95\\
\bottomrule
\end{tabular}
\par
\vspace{-0.75\skip\footins}
\renewcommand{\footnoterule}{}
\end{minipage}
\end{table}

\begin{figure}[H]
  \centering
  \includegraphics{plot2.pdf}
  \caption{Messwerte der Monozelle mit Gegenspannung mit linearer Ausgleichsgerade}
  \label{fig:klemmspannung_gegenspannung}
\end{figure}

\noindent Die Berechnungen für die Rechteck- und Sinusspannungsquelle erfolgen wieder mit Gleichung \eqref{eqn:uk}.
Die Messwerte für die Rechteckspannung sind in Tabelle \ref{tab:klemmspannung_rechteck}, die für die Sinusspannung in Tabelle \ref{tab:klemmspannung_sinus} zu finden.
Diese Ergebnisse wurden in Abbildung \ref{fig:klemmspannung_rechteck} aufgetragen.
Auch hier wird eine lineare Regression durchgeführt.
Entsprechendes für die Sinusspannung ist in Abbildung \ref{fig:klemmspannung_sinus} zu finden.
Damit ergibt sich für $R_i$ und $U_0$:

\begin{align*}
  \text{Rechteckspannung: }
    -a &= R_i = 61,0 \pm 1,7  \, \si{\ohm} \\
    b &= U_0 = 592,5 \pm 7,3 \, \si{\milli\volt} \\
  \text{Sinusspannung: }
  -a &= R_i = 711,26 \pm 4,92  \, \si{\ohm} \\
  b &= U_0 = 1089,14 \pm 2,66 \, \si{\volt} \\
\end{align*}

\begin{table}
\centering
\caption{Abhängigkeiten mit Rechteckausgang}
\label{tab:klemmspannung_rechteck}
\sisetup
{table-format=1.2}
\begin
{tabular}{S[table-format=3.0] S[table-format=3.5] S[table-format=3.2]}
\toprule
{$R/\,\si{\Omega}$} &{$I/\,\si{mA}$} &{$U_k/\,\si{mV}$} \\
\midrule
20.0 & 7.5 & 110\\
31.5 & 7.4 & 115\\
43.0 & 6.3 & 230\\
54.5 & 5.6 & 260\\
66.0 & 5.2 & 290\\
77.5 & 4.7 & 320\\
89.0 & 4.3 & 350\\
100.5 & 4.0 & 355\\
112.0 & 3.8 & 370\\
123.5 & 3.5 & 380\\
135.0 & 3.3 & 390\\
146.5 & 3.1 & 400\\
158.0 & 3.0 & 410\\
169.5 & 2.8 & 420\\
181.0 & 2.7 & 425\\
192.5 & 2.6 & 430\\
204.0 & 2.5 & 440\\
215.5 & 2.4 & 440\\
227.0 & 2.3 & 445\\
238.5 & 2.2 & 450\\
250.0 & 2.1 & 450\\
\bottomrule
\end{tabular}
\end{table}

\begin{figure}[h]
  \centering
  \includegraphics{plot3.pdf}
  \caption{Messwerte der Rechteckspannungsquelle mit linearer Ausgleichsgerade}
  \label{fig:klemmspannung_rechteck}
\end{figure}

\begin{table}
\centering
\caption{Abhängigkeiten mit Sinusausgang}
\label{tab:klemmspannung_sinus}
\sisetup
{table-format=1.2}
\begin{minipage}{4.5cm}
\begin
{tabular}{S[table-format=3.0] S[table-format=3.5] S[table-format=3.2]}
\toprule
{$R/\,\si{\Omega}$} &{$I/\,\si{mA}$} &{$U_k/\,\si{mV}$} \\
\midrule
100 & 1.11 & 300\\
345 & 1.08 & 320\\
590 & 0.99 & 380\\
835 & 0.78 & 530\\
1080 & 0.63 & 630\\
1325 & 0.57 & 690\\
1570 & 0.51 \footnote{Skalawechsel Amperemeter} & 740\\
1815 & 0.45 & 780\\
2060 & 0.40 & 810\\
2305 & 0.36 & 830\\
2550 & 0.35 & 850\\
2795 & 0.31 & 870\\
3040 & 0.29 & 880\\
3285 & 0.27 & 900\\
3530 & 0.25 & 910\\
3775 & 0.23 & 920\\
4020 & 0.21 & 940\\
4265 & 0.20 & 950\\
4510 & 0.18 & 955\\
4755 & 0.17 & 960\\
5000 & 0.16 & 970\\
\bottomrule
\end{tabular}
\par
\vspace{-0.75\skip\footins}
\renewcommand{\footnoterule}{}
\end{minipage}
\end{table}

\begin{figure}[H]
  \centering
  \includegraphics{plot4.pdf}
  \caption{Messwerte der Sinusspannungsquelle mit linearer Ausgleichsgerade}
  \label{fig:klemmspannung_sinus}
\end{figure}

\noindent Außer Acht gelassen wurde bisher, dass ein Voltmeter über einen endlichen Eingangswiderstand $R_V$ verfügt.
In den Berechnungen kommt es somit zu einem systematischen Fehler bei der entsprechenden Messung der Leerlaufspannung.
Die Gleichung \ref{eqn:klemmspannung} lässt sich dann zu

\begin{equation}
  U_0 = U_k + U_k\frac{R_i}{R_a}
\end{equation}

\noindent umstellen.
$U_k$ entspricht dabei dem im ersten Messvorgang aufgenommenen $U_0 = 1,4 \,\si{\volt}$.
$R_a$ hingegen dem Wert von $R_V = 10 \, \si{\mega\ohm}$.
Die Berechnung des Fehler erfolgt somit mit

\begin{align*}
  \Delta U_0 &= U_0 - U_k = 2,0538 \, \si{\micro\ohm} \\
  \frac{\Delta U_0}{U_k} &= 1,467 \cdot 10^{-4} \, \%
\end{align*}

\noindent Die systematischen Fehler sind somit sehr gering und werden auch im Folgenden nicht weiter beachtet. \\
Ein systematischer Fehler würde gemacht werden, sollte das Voltmeter noch hinter das Amperemeter geschalten werden.
Dies meint hinter den Punkt \textbf{H} in Abbildung \ref{fig:aufbaua}.
In diesem Fall würde das Voltmeter nicht nur den Spannungsabfall über der Spannungsquelle abgreifen, sondern auch den über dem Amperemeter.
Dies würde bedeuten, dass nicht mehr die reine Klemmspannung $U_k$ abgegriffen werden würde.

\newpage

\noindent Anhand der gemessenen Spannung $U_0$ und des Innenwiderstandes $R_i$ der Monozelle, lässt sich die am Lastwiderstand umgesetzte Leistung errechnen.
Die gemessenen Werte werden dabei mit dem theoretisch berechneten Verlauf verglichen:

\begin{equation}
  P = \frac{U_0^2}{(R_i + R_a)^2} R_a
\end{equation}

\begin{figure}[H]
  \centering
  \includegraphics{plot5.pdf}
  \caption{Umgesetzte Leistung am Lastwiderstand}
  \label{fig:leistung}
\end{figure}

\noindent Die zu beobachtenden Abweichungen von maximal ca. $0,007 \,\mathrm{V}$ liegen im Bereich der Messungenauigkeit.
