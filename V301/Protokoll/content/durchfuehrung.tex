\section{Durchführung}
\label{sec:Durchführung}

Zunächst wird die Leerlaufspannung $U_0$, der gegebenen Monozelle, gemessen.
Dafür wird lediglich ein Voltmeter mit dieser verbunden.
Die gemessene Spannung und der Eingangs-/ Innenwiderstand $R_V$ des Voltmeters werden dabei vermerkt.
Anschließend wird die Schaltung gemäß Abbildung \ref{fig:aufbaua} aufgebaut.
Das Voltmeter von vorher bleibt parallel geschaltet und soll weiterhin den Spannungsabfall dokumentieren.
$R_a$ verfügt hierbei über eine Spektrum von $10 - 20\,\si{\ohm}$.
Es sollten mindestens 20 Messpaare von der Klemmenspannung $U_k$ und des Belastungsstroms \textbf{I} genommen werden.

\begin{figure}
  \centering
  \begin{subfigure}{0.48\textwidth}
    \centering
    \begin{circuitikz}[line width=0.5pt]
      \draw(0,0)
      to[battery1,thick] (0,3) -- (0,4)
      to[short, i=$I$] (1,4)
      (1,4) to[short] (3,4)
      node[draw,circle,fill=white,minimum size=10pt,text=white,thick]{V}
      (3,4) to[short, -] (5,4);
      \draw(5,1.3) to[tgeneric] (5,4);
      \draw (5,0) to[short] (5,0.7)
      (5,0) to[short] (0,0)
      (2,0) to[short, *-] (2,2)
      node[draw,circle,fill=white,minimum size=10pt,text=white,thick]{V}
      (2,2) to[short, -*] (2,4)
      (4,4) to[short, *-] (4,3.5)
      (4,4) node[above] {H}
      (4.6,1) node[above,rotate=270,thick] {\huge$\top$}
      (2.3,2) node[right]{V}
      (2.8,4.2) node[above left]{A}
      (0.1,1.6) node[above left,thick]{+}
      (0,1.4) node[below left,thick]{-}
      (1,0) to[short,->] (1,0.7)
      (1,4) to[short,->] (1,1.3)
      (1,0.7) node[above] {$U_k$};
      \draw[->,thick]  (1.6,1.6) -- (2.4,2.4);
      \draw[->,thick]  (2.6,3.6) -- (3.4,4.4);
      \draw[->,thick]  (1,0.7) -- (1,0);
      \draw[->,thick]  (1,1.3) -- (1,4);
    \end{circuitikz}
    \caption{Messschaltung zur Bestimmung von $U_0$ und $R_i$}
    \label{fig:aufbaua}
  \end{subfigure}
  \hfill
  \begin{subfigure}{0.48\textwidth}
    \centering
    \begin{circuitikz}[line width=0.5pt]
      \draw(0,0)
      to[battery1,thick] (0,3) -- (0,4)
      to[short, i=$I$] (1,4)
      (1,4) to[short] (3,4)
      node[draw,circle,fill=white,minimum size=10pt,text=white,thick]{V}
      (3,4) to[short, -] (5,4);
      \draw(5,2.3) to[tgeneric] (5,4);
      \draw (5,0) to[dcvsource] (5,1.7)
      (5,0) to[short] (0,0)
      (2,0) to[short, *-] (2,2)
      node[draw,circle,fill=white,minimum size=10pt,text=white,thick]{V}
      (2,2) to[short, -*] (2,4)
      (4.6,2) node[above,rotate=270,thick] {\huge$\top$}
      (2.3,2) node[right]{V}
      (2.8,4.2) node[above left]{A}
      (0.1,1.6) node[above left,thick]{+}
      (0,1.4) node[below left,thick]{-}
      (4.7,1.1) node[above left,thick]{+}
      (4.6,0.6) node[below left,thick]{-}
      (1,0) to[short,->] (1,0.7)
      (1,4) to[short,->] (1,1.3)
      (1,0.7) node[above] {$U_k$};
      \draw[->,thick]  (1.6,1.6) -- (2.4,2.4);
      \draw[->,thick]  (2.6,3.6) -- (3.4,4.4);
      \draw[->,thick]  (1,0.7) -- (1,0);
      \draw[->,thick]  (1,1.3) -- (1,4);
    \end{circuitikz}
    \caption{wie Abbildung \ref{fig:aufbau}.\subref{fig:aufbaua} jedoch unter Verwendung einer Gegenspannung}
    \label{fig:aufbaub}
  \end{subfigure}
  \caption{Schaltbilder der im Versuch benutzten Aufbauten}
  \label{fig:aufbau}
\end{figure}

\noindent Nachfolgend wird die Schaltung gemäß Abbildung \ref{fig:aufbaub} modifiziert.
D.h., es wird nach dem Widerstand $R_a$ eine zusätzliche Gleichspannungsquelle angelegt.
Diese verfügt über die umgekehrte Polung der anderen Monozelle und weist etwa $2 \mathrm{V}$ mehr Spannung auf.
Danach wird wie oben bereits beschrieben, verfahren.
Der Widerstand $R_a$ wird erneut von $0-50 \si{\ohm}$ variiert und zwar so, dass erneut 20 Messwerte gemessen werden können.
Für die abschließenden Messungen wird die zweite Spannungsquelle wieder entfernt.
Die Schaltung ergibt sich also wieder wie in Abbildung \ref{fig:aufbaua}.
Allerdings wird statt einer Monozelle der Sinus-, bzw. Rechteck-ausgang eines RC- Generators verwendet.
Für den Rechteckausgang wird ein Widerstand $R_a$ genutzt, dessen Variatonsbereich bei $20-250 \si{\ohm}$ liegt.
Für die Messung muss an dem RC- Generator die Einstellung für $1 \si{\volt}$ ausgewählt werden.
Die Amplitude wird für die Rechteckspannung komplett aufgedreht.
Für die Messung werden erneut 20 Werte des Ampere- und Voltmeters abgelesen.
Dies wird ebenfalls für die Sinusspannung wiederholt.
Auch hier wird $1 \si{\volt}$ und die volle Sinusamplitude verwendet.
Der Widerstand $R_a$ verfügt nun allerdings über einen Variationsbereich von $0,1 - 5 \,\si{\kilo\ohm}$.
Es werden erneut 20 Messwerte des Ampere- und Voltmeters abgelesen.
