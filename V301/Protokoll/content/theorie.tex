\section{Theorie}
\label{sec:Theorie}

Als Spannungsquelle wird ein Gerät bezeichnet, das über einen bestimmten Zeitraum eine bestimmte elektrische Leistung liefert.
Es wird dabei zwischen (theoretischen) idealen- und realen Spannungsquellen unterschieden.
Während reale Spannungsquellen durch ihren Aufbau bedingt, über einen Innenwiderstand verfügen, werden ideale Spannungsquellen als innenwiderstandsfrei angesehen.
Spannungsquellen können deshalb in Schaltkreisen durch eine ideale Spannungsquelle $U$ und einen Innenwiderstand $R_i$ dargestellt werden.\\

\noindent Eine Spannungsquelle bzw. das Verhalten dieser in einem Stromkreis wird im Wesentlichen durch zwei Größen charakterisiert:

\subsection{Leerlaufspannung}
Als Leerlaufspannung, Quellspannung oder Urspannung $U_0$ bezeichnet man die an der unbelasteten Quelle abfallende Spannung.

\subsection{Innenwiderstand}
Der Innenwiderstand einer Spannungsquelle kommt dadurch zustande, dass Leitungen und Bauteile im Gerät keine idealen Leiter sind und sich so zu einem gesamt-Widerstand aufaddieren.
Wird ein Verbraucher mit Widerstand $R_a$ an die Spannungsquelle angeschlossen, fließt ein Strom und die an der Spannungsquelle gemessene Spannung verändert sich.
Diese, so genannte Klemmenspannung $U_k$, einer belasteten Spannungsquelle ist immer kleiner als die Leerlaufspannung $U_0$.
Dieses Phänomen lässt sich leicht mit dem 2. Kirchhoffschen Gesetz (Maschenregel) erklären:

\begin{figure}[h]
\centering
\begin{circuitikz}
 \draw (0,0)
 to[dcvsource, v=$U_0$] (0,2)
 to[R=$R_i$] (3,2)
 to[short, i=$I$] (6,2)
 to[R=$R_a$] (6,0)
 to[short] (0,0);
 \draw[->]  (3,0) -- node[left] {$U_k$} (3,2);
 \draw[->]  (3,2) -- node[left] {} (3,0);
\end{circuitikz}
\caption{Spannungsquelle mit Lastwiderstand und Klemmspannung $U_k$}
\end{figure}

\begin{align}
  &U_0 = \sum_{k=1}^{n} U_k \implies U_0 = I R_i + I R_a \\
  \implies &U_k = U_0 - I R_i \label{eqn:uk}
\end{align}

\noindent Anhand von Gleichung \eqref{eqn:uk} wird sofort klar, warum die Klemmspannung bei einem angeschlossenen Verbraucher sinken muss.
Je größer der Strom $I$, der durch $R_i$ fließt, desto kleiner wird $U_k$. Aus diesem Grund verfügen Spannungsmessgeräte über einen möglichst großen Innenwiderstand.
Wird ein solches Messgerät an eine Spannungsquelle angeschlossen, enstpricht die gemessene Spannung nahezu der Leerlaufspannung aufgrund des sehr hohen Innenwiderstandes.

\subsection{Leistungsanpassung}
Als Leistungsanpassung bezeichnet man den Zustand eines Stromkreises, in dem die maximale Leistung der Spannungsquelle im Verbraucher umgesetzt wird.
Dies kann durch die Variation des Widerstandes $R_a$ erreicht werden. Für die am Verbraucher umgesetzte Leistung gilt \cite[S.2]{anleitung}:

\begin{equation}
  P_a = I^2 R_a
\end{equation}

\noindent Dies ergibt einen Graphen mit einem Maximum $P_{\text{max}}$, das genau dann erreicht wird, wenn Innenwiderstand der Quelle und der Widerstand des Verbrauchers gleich groß sind.
Der Wirkungsgrad beträgt dann genau $\eta = 0,5$. Falls der Innenwiderstand viel größer als der Widerstand des Verbrauchers ist, ist die Klemmenspannung sehr niedrig und es wird kaum Leistung an $R_a$ umgesetzt.
Im umgekehrten Fall fließt kaum Strom durch $R_a$ was ebenfalls zu einer geringen umgesetzten Leistung führt. \\

\noindent In der Nachrichten- und Messtechnik wird die Leistungsanpassung oft verwendet, z.B. beim Bau von Antennen um Signale optimal übertragen zu können.
In der Starkstromtechnik hingegegen ist diese Methode aufgrund eines Wirkungsgrades von $0,5$ nicht praktikabel, da es dort bei sehr hohen Strömen zu extremen thermischen Verlusten kommt.

\subsection{Generatoren}

Das Konzept des Innenwiderstandes findet nicht nur bei Spannungsquellen, sondern auch bei elektrischen Generatoren Anwendung.
Da der Widerstand hierbei durch Rückkopplungsmechanismen festgelegt wird, wird von der einfachen Darstellung als ohmscher Widerstand zu einer differentiellen Darstellung übergegangen:

\begin{equation}
  R_i = \frac{dU_k}{dI}
\end{equation}

\section{Fehlerrechung}
\label{sec:Fehlerrechnung}

Ausgleichsgeraden werden nach Python mit

\begin{align}
  y &= a \cdot x + b \label{eqn:ausgleich}\\
  a &= \frac{\overline{xy}-\overline{x} \, \overline{y}}{\overline{x^2}-\overline{x}^2} \\
  b &= \frac{\overline{x^2}\overline{y}-\overline{x} \, \overline{xy}}{\overline{x^2}-\overline{x}^2}
\end{align}

\noindent berechnet. Die Regressionen, Ausgleichsgeraden und Fehler-Berechnungen werden mithilfe von Numpy, matplotlib und scipy durchgeführt.
