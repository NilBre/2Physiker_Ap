\section{Durchführung}
\label{sec:Durchführung}

\subsection{Zeitkonstante bestimmen}

\begin{figure}[h]
\centering
\begin{circuitikz}
 \draw (0,0)
 to[sqV, l=Rechteckgenerator] (0,2)
 to[R=$R$] (4,2)
 to[C=$C$,*-*] (4,0)
 to[short] (0,0);
 \draw (4,2)
 to[short] (8,2)
 to[SCOPE, l=CH1, *-*] (8,0)
 to[short] (4,0);
\end{circuitikz}
\caption{Messschaltung für Bestimmung der Differenzspannung $\Delta U$}
\label{fig:fig1}
\end{figure}

Um die Zeitkonstante $\tau$ experimentell zu bestimmen, werden ein Rechteckgenerator und ein Oszilloskop wie in \ref{fig:fig1} beschrieben
an einen Kodensator angeschlossen. Nach der korrekten Justierung des Triggers und der Eichung der Achsen am Oszilloskop werden in mehreren
Messungen jeweils die Zeit $t$ und die über dem Kondensator abfallende Spannung $\Delta U$ notiert.

\subsection{Messung der Amplitude}

\begin{figure}[h]
\centering
\begin{circuitikz}
 \draw (0,0)
 to[sV, l=$U_0$] (0,2)
 to[R=$R$] (4,2)
 to[C=$C$,*-*] (4,0)
 to[short] (0,0);
 \draw (4,2)
 to[short] (8,2)
 to[SCOPE, l=CH1, *-*] (8,0)
 to[short] (4,0);
\end{circuitikz}
\caption{Messschaltung für Bestimmung der Kondensatorspannungsamplitude $\Delta U$}
\label{fig:fig2}
\end{figure}

Um die Amplitude der Kondensatorspannung $U_c$ zu vermessen wird eine Schaltung entsprechend \ref{fig:fig2} aufgebaut. Am Sinusgenerator
werden Frequenzen von $10\,\mathrm{Hz}$ bis $20\,000\,\mathrm{Hz}$ eingestellt und die jeweiligen Amplituden am Oszilloskop ausgelesen und
notiert. Durch Messung der Spannungsamplitude $U_0$ kann bestätigt werden, dass die Spannung wie erwartet unabhängig von der Frequenz
$\nu$ ist.

\subsection{Phasenverschiebung}

\begin{figure}[h]
\centering
\begin{circuitikz}
 \draw (0,0)
 to[sV, l=$U_0$] (0,2)
 to[R=$R$] (4,2)
 to[C=$C$,*-*] (4,0)
 to[short] (0,0);
 \draw (4,2)
 to[short] (8,2)
 to[SCOPE, l=CH1, *-*] (8,0)
 to[short] (4,0);
 \draw (0,2)
  to[short,*-] (0,3)
  to[short] (11,3)
  to[SCOPE, l=CH2, *-*] (11,0)
  to[short] (8,0);
\end{circuitikz}
\caption{Phasenverschiebung zwischen Eingangs- und Kondensatorspannung}
\label{fig:fig3}
\end{figure}

Für die Messung der enstehenden Phasenverschiebung $\Delta \phi$ zwischen Eingangsspannung $U_0$ und Kondensatorspannung $U_c$ wird ein Zweikanaloszilloskop wie in \ref{fig:fig3} beschrieben eingesetzt. In mehreren Messdurchgängen von $10\,\mathrm{Hz}$ bis $5\,000\,\mathrm{Hz}$ werden jeweils die Zeitdifferenz der Nulldurchgänge der beiden Schwingungen, sowie die Schwingungsdauer mit dem Oszilloskop vermessen und notiert.

\subsection{Integrator}
Um die Funktionalität des RC-Kreises als Integrator zu untersuchen wird wieder die Schaltung mit dem Zweikanaloszilloskop gewählt. (vgl. \ref{fig:fig3}) Die Frequenz des Frequenzgenerators wird passend gewählt ($\omega \gg \frac{1}{RC}$) und es werden verschiedene Spannungen (Rechteck, Sinus, Dreieck) erzeugt. Auf dem Oszilloskop sind dann jeweils die integrierte sowie die integrierende Spannungskurve sichtbar. Für die Auswertung wird ein Screenshot abgespeichert.
