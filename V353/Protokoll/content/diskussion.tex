\section{Diskussion}
\label{sec:Diskussion}
In Abbildung \ref{fig:zeitkonstante} ist zu sehen, dass die gemessenen Werte nur bedingt einem linearen Verlauf folgen.
Unser Verlauf legt eher einen anderen Funktionstypus nahe.
Es ist davon auszugehen, dass der Messbereich von lediglich $500 \, \mathrm{\mu s}$ zu klein gewählt wurde und deshalb nur ein sehr kleiner Ausschnitt der Entladekurve dokumentiert wurde.
Bei der Auswertung wurde deshalb eine Auswahl an Messpunkten gewählt, die es ermöglichen, einen linearen Fit durchzuführen ohne dabei große Fehler in Kauf nehmen zu müssen. Zwar ist immer noch eine Abweichung zu den anderen errechneten Zeitkonstanten vorhanden, durch die geschickte Auswahl der Messwerte konnte allerdings eine bessere Annäherung erreicht werden.
\\ \\
Der Fit für die Werte in Abbildung \ref{fig:amplitude} hingegen gibt das Verhältnis der Werte sehr gut wieder.
Dies erklärt auch die hohe Genauigkeit der errechneten RC- Konstante.
Lediglich am Anfang der Messung sind Abweichungen erkennbar.
So wird, wie in Tabelle \ref{tab:amplitude} zu sehen, der Spannungswert $U_c$ für die zwei nachfolgenden Frequenzen größer.
Dies ist vermutlich auf die Sensibilität der Messgeräte in diesen Frequenzbereichen zurückzuführen.
Es ließ sich für diese ersten drei Frequenzen oft nur schwer Messwerte ablesen, da die Werte teilweise sprangen.
\\ \\
Der Wert für RC, den wir der Phasenverschiebung entnommen haben, ist etwas ungenauer.
Dies liegt daran, dass unser Fit, in Abbildung \ref{fig:phase}, mit recht wenig Werten auskommen musste.
An dieser Stelle wäre es sinnvoll gewesen noch weitere Werte für unterschiedliche Frequenzen zu messen.
\\ \\
Die gemessenen Werte für den Polarplot in Tabelle \ref{tab:polar} sind sehr nahe an den Werten der Theoriekurve, wie man in Abbildung \ref{fig:polar} erkennen kann.
\\ \\
Alles in allem ist es auffällig, dass für 3 verschiedene Ermittlungsverfahren für RC jeweils andere Werte errechnet wurden.
Theoretisch lässt sich dies nur auf systematische Fehler zurückführen.
Die einzelnen Verfahren sind nicht unbedingt gleich gut geeignet, da sich durch die verwendeten Geräte unterschiedliche Parameter, wie der Generatorinnenwiderstand, ergeben können.
Hinzu kommen unterschiedlice Genauigkeiten beim Messen der Werte und die entsprechende Fehlerfortpflanzung.
Welcher Wert allerdings dem tatsächlichen Wert am nächsten liegt, konnte nicht errechnet werden, da dafür die benötigten Angaben der Komponenten fehlten.
