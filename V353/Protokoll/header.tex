\documentclass[
  bibliography=totoc,     % Literatur im Inhaltsverzeichnis
  captions=tableheading,  % Tabellenüberschriften
  titlepage=firstiscover, % Titelseite ist Deckblatt
]{scrartcl}

% Paket float verbessern
\usepackage{scrhack}

% Warnung, falls nochmal kompiliert werden muss
\usepackage[aux]{rerunfilecheck}

% unverzichtbare Mathe-Befehle
\usepackage{amsmath}
% viele Mathe-Symbole
\usepackage{amssymb}
% Erweiterungen für amsmath
\usepackage{mathtools}
% Stromkreise zeichnen
\usepackage{tikz}
\usepackage[european]{circuitikz}

% Fonteinstellungen
\usepackage{fontspec}
% Latin Modern Fonts werden automatisch geladen
% Alternativ zum Beispiel:
%\setromanfont{Libertinus Serif}
%\setsansfont{Libertinus Sans}
%\setmonofont{Libertinus Mono}

% Wenn man andere Schriftarten gesetzt hat,
% sollte man das Seiten-Layout neu berechnen lassen
\recalctypearea{}

% deutsche Spracheinstellungen
\usepackage{polyglossia}
\setmainlanguage{german}


\usepackage[
  math-style=ISO,    % ┐
  bold-style=ISO,    % │
  sans-style=italic, % │ ISO-Standard folgen
  nabla=upright,     % │
  partial=upright,   % ┘
  warnings-off={           % ┐
    mathtools-colon,       % │ unnötige Warnungen ausschalten
    mathtools-overbracket, % │
  },                       % ┘
]{unicode-math}

% traditionelle Fonts für Mathematik
\setmathfont{Latin Modern Math}
% Alternativ zum Beispiel:
%\setmathfont{Libertinus Math}

\setmathfont{XITS Math}[range={scr, bfscr}]
\setmathfont{XITS Math}[range={cal, bfcal}, StylisticSet=1]

% Zahlen und Einheiten
\usepackage[
  locale=DE,                   % deutsche Einstellungen
  separate-uncertainty=true,   % immer Fehler mit \pm
  per-mode=symbol-or-fraction, % / in inline math, fraction in display math
]{siunitx}

% chemische Formeln
\usepackage[
  version=4,
  math-greek=default, % ┐ mit unicode-math zusammenarbeiten
  text-greek=default, % ┘
]{mhchem}

% richtige Anführungszeichen
\usepackage[autostyle]{csquotes}

% schöne Brüche im Text
\usepackage{xfrac}

% Standardplatzierung für Floats einstellen
\usepackage{float}
\floatplacement{figure}{htbp}
\floatplacement{table}{htbp}

% Floats innerhalb einer Section halten
\usepackage[
  section, % Floats innerhalb der Section halten
  below,   % unterhalb der Section aber auf der selben Seite ist ok
]{placeins}

% Seite drehen für breite Tabellen: landscape Umgebung
\usepackage{pdflscape}

% Captions schöner machen.
\usepackage[
  labelfont=bf,        % Tabelle x: Abbildung y: ist jetzt fett
  font=small,          % Schrift etwas kleiner als Dokument
  width=0.9\textwidth, % maximale Breite einer Caption schmaler
]{caption}
% subfigure, subtable, subref
\usepackage{subcaption}

% Grafiken können eingebunden werden
\usepackage{graphicx}
% größere Variation von Dateinamen möglich
\usepackage{grffile}

% schöne Tabellen
\usepackage{booktabs}

% Verbesserungen am Schriftbild
\usepackage{microtype}

% Literaturverzeichnis
\usepackage[
  backend=biber,
]{biblatex}
% Quellendatenbank
\addbibresource{lit.bib}
\addbibresource{programme.bib}

% Hyperlinks im Dokument
\usepackage[
  unicode,        % Unicode in PDF-Attributen erlauben
  pdfusetitle,    % Titel, Autoren und Datum als PDF-Attribute
  pdfcreator={},  % ┐ PDF-Attribute säubern
  pdfproducer={}, % ┘
]{hyperref}
% erweiterte Bookmarks im PDF
\usepackage{bookmark}

% Trennung von Wörtern mit Strichen
\usepackage[shortcuts]{extdash}

% Nutzung von ToDo- Notes
\usepackage{todonotes}

\author{%
  Fabian Koch\\%
  \href{mailto:fabian3.koch@udo.edu}{fabian3.koch@udo.edu}%
  \texorpdfstring{\and}{,}%
  Raphael Sütterlin\\%
  \href{mailto:raphael.suetterlin@udo.edu}{raphael.suetterlin@udo.edu}%
}
\publishers{TU Dortmund – Fakultät Physik}

%Oszilloskop Symbol für circuitikz%
\makeatletter
\def\TikzBipolePath#1#2{\pgf@circ@bipole@path{#1}{#2}}

\newlength{\ResUp} \newlength{\ResDown}
\newlength{\ResLeft} \newlength{\ResRight}
\ctikzset{bipoles/SCOPE/height/.initial=.60}
\ctikzset{bipoles/SCOPE/width/.initial=.60}
\pgfcircdeclarebipole{}
{\ctikzvalof{bipoles/SCOPE/height}}
{SCOPE}
{\ctikzvalof{bipoles/SCOPE/height}}
{\ctikzvalof{bipoles/SCOPE/width}}
{
\pgfsetlinewidth{\pgfkeysvalueof{/tikz/circuitikz/bipoles/thickness}\pgfstartlinewidth}
  \pgfextracty{\ResUp}{\northeast}    %ResUp make usable
  \pgfextractx{\ResRight}{\northeast} %ResRight make usable
  \pgfextractx{\ResLeft}{\southwest}  %ResLeft make usable
  \pgfextracty{\ResDown}{\southwest}  %ResDown make usable
  \def\pgfcircmathresult{\expandafter\pgf@circ@stripdecimals\pgf@circ@direction\pgf@nil}
      \ifnum \pgfcircmathresult > 45 \ifnum \pgfcircmathresult < 135
          \pgftransformrotate{270}
      \fi\fi
      \ifnum \pgfcircmathresult > 135 \ifnum \pgfcircmathresult < 225
          \pgftransformrotate{180}
      \fi\fi
      \ifnum \pgfcircmathresult > 225 \ifnum \pgfcircmathresult < 315
          \pgftransformrotate{90}
      \fi\fi
  \pgfpathmoveto{\pgfpoint{0.75\ResLeft}{0.25\ResDown}}
  \pgfpathlineto{\pgfpoint{0.05\ResLeft}{0.25\ResUp}}
  \pgfpathlineto{\pgfpoint{0.05\ResLeft}{0.25\ResDown}}
  \pgfpathlineto{\pgfpoint{0.65\ResRight}{0.25\ResUp}}
  \pgfpathlineto{\pgfpoint{0.65\ResRight}{0.25\ResDown}}
  \pgfpathlineto{\pgfpoint{0.65\ResRight}{0.25\ResDown}}
  \pgfpathmoveto{\pgfpoint{1.25\ResLeft}{0.5\ResDown}}
  \pgfpathlineto{\pgfpoint{1.75\ResLeft}{0.5\ResDown}}
  \pgfpathmoveto{\pgfpoint{1.5\ResLeft}{0.25\ResDown}}
  \pgfpathlineto{\pgfpoint{1.5\ResLeft}{0.75\ResDown}}
  \pgfpathmoveto{\pgfpoint{1.25\ResRight}{0.75\ResDown}}
  \pgfpathlineto{\pgfpoint{1.75\ResRight}{0.75\ResDown}}
  \pgfpathmoveto{\pgfpoint{1.5\ResRight}{0.45\ResDown}}
  \pgfpathlineto{\pgfpoint{1.5\ResRight}{0.75\ResDown}}

  \pgfpathellipse{\pgfpointorigin}{\pgfpoint{0}{\ResUp}}{\pgfpoint{\ResLeft}{0}}
  \pgfusepath{draw}
}
\def\SCOPEpath#1{\TikzBipolePath{SCOPE}{#1}}
\tikzset{SCOPE/.style = {\circuitikzbasekey, /tikz/to path=\SCOPEpath, l=#1}}
\makeatother
